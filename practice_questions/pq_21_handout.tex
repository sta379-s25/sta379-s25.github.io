\documentclass[11pt]{article}
\usepackage{url}
\usepackage{alltt}
\usepackage{bm}
\usepackage{bbm}
\linespread{1}
\textwidth 6.5in
\oddsidemargin 0.in
\addtolength{\topmargin}{-1in}
\addtolength{\textheight}{2in}

\usepackage{amsmath}
\usepackage{amssymb}
\usepackage{hyperref}
\usepackage{bbm}

\newcommand{\indep}{\perp \!\!\! \perp}

\begin{document}


\begin{center}
\Large
Activity: Motivating Gaussian quadrature \\
\normalsize
\vspace{5mm}
\end{center}

\noindent \textbf{Instructions:} Submit your work as a single PDF. You may choose to either hand-write your work and submit a PDF scan, or type your work using LaTeX and submit the resulting PDF. See the course website for a \href{https://sta711-s25.github.io/homework/hw_template.tex}{homework template file} and \href{https://sta711-s25.github.io/homework/latex_instructions/}{instructions} on getting started with LaTeX and Overleaf.

\section*{Part 1}

Suppose we observe $n$ points $(x_1, y_1),...,(x_n, y_n)$, and let

$$L_{n,i}(x) = \prod \limits_{k : k \neq i} \frac{(x - x_k)}{(x_i - x_k)}$$

\noindent This function $L_{n,i}(x)$ is a polynomial, and it turns out that $L_{n,i}(x)$ plays an important role in deriving Gaussian quadrature. To begin, let's explore some properties of $L_{n,i}(x)$.

\begin{enumerate}
\item Show that $L_{n,i}(x_i) = 1$

\vspace{3cm}

\item Show that $L_{n,i}(x_k) = 0$ for all $k \neq i$

\vspace{3cm}
\end{enumerate}

\noindent Now let

$$q(x) = \sum \limits_{i=1}^n y_i L_{n,i}(x)$$

\noindent $q(x)$ is also a polynomial.

\begin{enumerate}
\item[3.] Using the results from questions 1 and 2, calculate $q(x_1),...,q(x_n)$.
\end{enumerate}

\newpage

\subsection*{Plotting $q(x)$}

The following code provides a function `q` to plot $q(x)$ between -1 and 1:

\begin{verbatim}
# calculate q at a single point
# x: point to evaluate q(x)
# xi: the points x1,...,xn
# yi: the points y1,...,yn
q_helper <- function(x, xi, yi){
  lp <- sapply(1:length(xi), 
               function(i){prod((x - xi[-i])/(xi[i] - xi[-i]))})
  sum(yi*lp)
}

# calculate q at a vector of new points
# x: point to evaluate q(x)
# xi: the points x1,...,xn
# yi: the points y1,...,yn
q <- function(x, xi, yi){
  sapply(x, function(t){q_helper(t, xi, yi)})
}


xi <- seq(-1, 1, length.out = 5)
yi <- xi^3
plot(xi, yi, pch=16)

x <- seq(-1, 1, 0.01)
lines(x, q(x, xi, yi))
\end{verbatim}

\begin{enumerate}
\item[4.] Run the code to add $q(x)$ to the plot with the five points $(x_1, y_1),...,(x_n, y_n)$. What do you notice about $q(x)$?

\vspace{3cm}

\item[5.]  To your plot from question 4, add the curve $y = x^3$ (the original function from which the $(x_i, y_i)$ were sampled). Comment on $q(x)$ vs. $x^3$.
\end{enumerate}

\newpage

\subsection*{Another example}

The following code samples $n=4$ points $(x_1, y_1),...,(x_n, y_n)$ from the 7th degree polynomial

$$f(x) = 10(x^7 - 1.6225x^5 +0.79875x^3 - 0.113906x)$$
and plots both the true polynomial $f(x)$ (in red) and the polynomial $q(x)$ (in black):

\begin{verbatim}
f <- function(x){
  10*(x^7 -1.6225*x^5 +0.79875*x^3 - 0.113906*x)
}

n <- 4
xi <- seq(-1, 1, length.out=n)
yi <- f(xi)

plot(xi, yi, pch=16, xlab="x", ylab="y")

x <- seq(-1, 1, 0.01)
lines(x, q(x, xi, yi))
lines(x, f(x), col="red")
\end{verbatim}

\begin{enumerate}
\item[6.] Comment on $q(x)$ vs. $f(x)$.

\vspace{2cm}

\item[7.] Now rerun the code with $n=5, 6, 7,$ and $8$ nodes. For each $n$, compare $q(x)$ to $f(x)$.

\vspace{3cm}
\end{enumerate}

\subsection*{Key points}

\begin{enumerate}
\item[8.] What does the function $q(x)$ do?

\vspace{2cm}

\item[9.] Why is the number of points $n$ important?
\end{enumerate}

\newpage

\section*{Part 2}

Previously in class, we found that the "best" two-point rule to approximate the integral of $f$ was

$$\int \limits_{-1}^1 f(x) dx \approx w_1 f(x_1) + w_2 f(x_2)$$
with $x_1 = -1/\sqrt{3}$, $x_2 = 1/\sqrt{3}$, and $w_1 = w_2 = 1$. 

\noindent Where do these weights come from? By using the polynomial interpolation $q(x) = \sum \limits_{i=1}^n f(x_i) L_{n,i}(x)$, we argued that

$$w_i = \int \limits_{-1}^1 L_{n,i}(x) dx$$

\begin{enumerate}
\item[10.] For the two-point rule, we have points $L_{2,1}(x) = \dfrac{x - x_2}{x_1 - x_2}$ with $x_1 = -1/\sqrt{3}$, $x_2 = 1/\sqrt{3}$. Show that

$$\int \limits_{-1}^1 L_{2,1}(x) dx = 1$$
\end{enumerate}

\end{document}
