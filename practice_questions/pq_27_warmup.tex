\documentclass[12pt]{article}
\usepackage{url}
\usepackage{alltt}
\usepackage{bm}
\usepackage{bbm}
\linespread{1}
\textwidth 6.5in
\oddsidemargin 0.in
\addtolength{\topmargin}{-1in}
\addtolength{\textheight}{2in}

\usepackage{amsmath}
\usepackage{amssymb}
\usepackage{hyperref}
\usepackage{bbm}

\newcommand{\indep}{\perp \!\!\! \perp}

\begin{document}


\begin{center}
\Large
Warmup: Variance reduction in Monte Carlo integration \\
\normalsize
\vspace{5mm}
\end{center}

\noindent \textbf{Group members:}

\section*{Monte Carlo integration}

Suppose we wish to approximate the integral
$$\theta = \int \limits_{-\infty}^\infty \frac{x}{2^x - 1} \frac{1}{\sqrt{2\pi}}e^{-\frac{1}{2}x^2} dx = \mathbb{E}[g(X)]$$
where $X \sim N(0, 1)$ and $g(x) = \dfrac{x}{2^x - 1}$.\\

\noindent The basic Monte Carlo approach is to do the following:

\begin{itemize}
\item Sample $X_1,...,X_n \overset{iid}{\sim} N(0, 1)$
\item $\widehat{\theta}_1 = \frac{1}{n} \sum \limits_{i=1}^n g(X_i)$
\end{itemize}

\subsection*{Questions}

\begin{enumerate}
\item Use Monte Carlo integration with $n = 1000$ to estimate $\theta$. Repeat the process many times to approximate $\mathbb{E}[\widehat{\theta}_1]$ and $Var(\widehat{\theta}_1)$. Write down the approximate values of $\mathbb{E}[\widehat{\theta}_1]$ and $Var(\widehat{\theta}_1)$.

\end{enumerate}

\vspace{4cm}

\section*{A different Monte Carlo estimator}

Now consider the estimator
$$\widehat{\theta}_2 = \frac{1}{n} \sum \limits_{i=1}^{n/2} (g(X_i) + g(-X_i))$$
where $X_1,...,X_{n/2} \sim N(0, 1)$ as before.

\begin{enumerate}
\item[2.] Use $n = 1000$ (so $n/2 = 500$) to compute this second estimator $\widehat{\theta}_2$. Repeat the process many times to approximate $\mathbb{E}[\widehat{\theta}_2]$ and $Var(\widehat{\theta}_2)$.

\vspace{4cm}

\item[3.] What is the percent reduction in variance of $\widehat{\theta}_2$ compared to $\widehat{\theta}_1$? Remember that the percent reduction in variance is calculated by

$$100 \cdot\dfrac{Var(\widehat{\theta}_1) - Var(\widehat{\theta}_2)}{Var(\widehat{\theta}_1)}$$

\vspace{4cm}

\item[4.] Using R, estimate the correlation $cor(g(X_i), g(-X_i))$ with $X_i \sim N(0, 1)$ and $g$ given as above (the \texttt{cor} function will be helpful). How is the correlation related to the percent reduction in variance?
\end{enumerate}


\end{document}
