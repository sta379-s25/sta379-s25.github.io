\documentclass[12pt]{article}
\usepackage{url}
\usepackage{alltt}
\usepackage{bm}
\usepackage{bbm}
\linespread{1}
\textwidth 6.5in
\oddsidemargin 0.in
\addtolength{\topmargin}{-1in}
\addtolength{\textheight}{2in}

\usepackage{amsmath}
\usepackage{amssymb}
\usepackage{hyperref}
\usepackage{bbm}

\newcommand{\indep}{\perp \!\!\! \perp}

\begin{document}


\begin{center}
\Large
Warmup: antithetic sampling \\
\normalsize
\vspace{5mm}
\end{center}

\noindent \textbf{Group members:}

\section*{Monte Carlo integration}

Suppose we wish to approximate the integral

$$\theta = \int \limits_0^1 e^x dx = \mathbb{E}[g(U)]$$

\noindent where $U \sim Uniform(0, 1)$ and $g(x) = e^x$.

\subsection*{Simple Monte Carlo estimate}

The simple Monte Carlo estimate does the following:

\begin{enumerate}
\item Sample $U_1,...,U_n \overset{iid}{\sim} Uniform(0, 1)$

\item $\widehat{\theta}_{MC} = \frac{1}{n} \sum \limits_{i=1}^n g(U_i)$
\end{enumerate}

\noindent Here is code to approximate the variance of the simple MC estimate with $n = 1000$:

\begin{verbatim}
g <- function(x){
  exp(x)
}

n <- 1000
nsim <- 1000

theta_hat_mc <- rep(NA, nsim)
for(i in 1:nsim){
  u <- runif(n)
  theta_hat_mc[i] <- mean(g(u))
}

var(theta_hat_mc)
\end{verbatim}

\subsection*{Antithetic sampling}

One method for reducing the variance of a Monte Carlo estimate is with \textbf{antithetic sampling}:

\begin{itemize}
\item Sample $U_1,...,U_{n/2} \sim Uniform(0, 1)$

\item $\widehat{\theta}_{AS} = \frac{1}{n} \sum \limits_{i=1}^{n/2} (g(U_i) + g(1 - U_i))$
\end{itemize}

\newpage

\subsection*{Questions}

\begin{enumerate}
\item Using R, approximate the variance of the antithetic estimator $\widehat{\theta}_{AS}$ with $n = 1000$, and compute the percent reduction in variance compared to the simple estimator $\widehat{\theta}_{MC}$.

\vspace{6cm}

\item Using R, approximate $\rho = Cor(g(U), g(1 - U))$ and compare to the percent reduction in variance from question 1.
\end{enumerate}


\end{document}
